% Inbuilt themes in beamer
\documentclass{beamer}

% Theme choice:
\usetheme{CambridgeUS}

% Title page details: 
\title{Assignment 6} 
\author{Beeram Sandya cs21btech11006}
\date{\today}
\logo{\large \LaTeX{}}


\begin{document}

% Title page frame
\begin{frame}
    \titlepage 
\end{frame}

% Remove logo from the next slides
\logo{}


% Outline frame
\begin{frame}{Outline}
    \tableofcontents
\end{frame}

% Lists frame
\section{Question}
\begin{frame}{Question}
\begin{block}{Papoulis ch7 Ex 7.27}
An infinite sum is by definition a limit :
\[\sum_{k=1}^{\infty} x_k = \lim_{n\to\infty} y_n \] 
\[y_n = \sum_{k=1}^{\infty} x_k\]
Show that if the random variables $x_k$ are independent with zero mean and variance 
$\sigma_k^2$, then the sum exists in the MS sense if and only if 
\[\sum_{k=1}^{\infty} \sigma_k^2 < \infty\] 

\end{block}
\end{frame}

% Blocks frame
\section{Solution}
\begin{frame}{Solution}
Given mean of random variable is zero $=> E\{x_k\}=0$ \\
\[E\{x_k^2\} = \sigma_k^2\]

\[E\{(\Sigma x_k)^2\} = \Sigma E\{x_k^2\} = \Sigma \sigma_k^2\] 
If $\sum_{k=1}^{\infty} \sigma_k^2 < \infty$then given $\epsilon > 0$ we can find $n_o$ such that \[\sum_{k=n+1}^{n+m} \sigma_k^2 < \epsilon \] for any m and $n>n_o$
.Thus,
\[E\{(y_{m+n}-y_n)^2\} = E\{(\sum_{k=n+1}^{m+n} x_k)^2\} = \sum_{k=n+1}^{n+m} \sigma_k^2 < \epsilon\]
This shows that (from Cauchy), $y_k$ converges in the MS sense.

\end{frame} 

\end{document}